\documentclass[12pt]{article}

\usepackage{fullpage}
\usepackage{multicol,multirow}
\usepackage{tabularx}
\usepackage{ulem}
\usepackage[utf8]{inputenc}
\usepackage[russian]{babel}
\usepackage{amsmath}
\usepackage{amssymb}


\usepackage{amsmath}
\usepackage{amssymb}
\usepackage{hyperref}
\usepackage{longtable}
\usepackage[table]{xcolor}
\usepackage{array}
\usepackage{color}
\usepackage{xcolor}
\usepackage{graphicx}
\DeclareGraphicsExtensions{.pdf,.png,.jpg}

\definecolor{lightcorn}{RGB}{228,160,16}

\usepackage{tikz}  
\usetikzlibrary{graphs}

\usepackage{hyperref}
%\usepackage[usenames,dvipsnames,svgnames,table,rgb]{xcolor}
\hypersetup{				% Гиперссылки
	%	unicode=true,           % русские буквы в раздела PDF
	%	pdftitle={Заголовок},   % Заголовок
	%	pdfauthor={Автор},      % Автор
	%	pdfsubject={Тема},      % Тема
	%	pdfcreator={Создатель}, % Создатель
	%	pdfproducer={Производитель}, % Производитель
	%	pdfkeywords={keyword1} {key2} {key3}, % Ключевые слова
	colorlinks=true,       	% false: ссылки в рамках; true: цветные ссылки
	linkcolor=lightcorn,          % внутренние ссылки
	%	citecolor=green,        % на библиографию
	%	filecolor=magenta,      % на файлы
	urlcolor=blue           % на URL
}

\definecolor{lightgray}{gray}{0.9}
\usepackage{minted}


\setminted[]{
	breaklines=true,
	tabsize=2,
	mathescape,
	frame=lines,
	framesep=2mm,
	bgcolor=lightgray,
	fontsize=\footnotesize,
	baselinestretch=1.1,
	numbers=left,
	linenos
}



\usepackage{titlesec}

\titleformat{\section}
  {\normalfont\Large\bfseries}{\thesection.}{0.3em}{}

\titleformat{\subsection}
  {\normalfont\large\bfseries}{\thesubsection.}{0.3em}{}

\titlespacing{\section}{0pt}{*2}{*2}
\titlespacing{\subsection}{0pt}{*1}{*1}
\titlespacing{\subsubsection}{0pt}{*0}{*0}
\usepackage{listings}
\lstloadlanguages{Lisp}
\lstset{extendedchars=false,
	breaklines=true,
	breakatwhitespace=true,
	keepspaces = true,
	tabsize=2
}
\begin{document}

\begin{center}
	\Large{\textbf{Отчет по лабораторной работе \textnumero 1 
		по курсу \\ \guillemotleft Искусственный Интелект\guillemotright}}
	
\end{center}
\begin{flushright}

Студент группы 8О-306 МАИ \textit{Недосеков Иван}, \textnumero 18 по списку \\
\makebox[7cm]{Контакты: {\tt ivan-nedd@mail.ru} \hfill} \\
\makebox[7cm]{Работа выполнена: \today \hfill} \\
\ \\
\makebox[7cm]{Преподаватель: Самиров А. \hfill} \\
\makebox[7cm]{Отчет сдан: \hfill} \\
\makebox[7cm]{Итоговая оценка: \hfill} \\
\makebox[7cm]{Подпись преподавателя: \hfill} \\

\end{flushright}

\section{Тема работы}
Обработка данных

\section{Цель работы}
Формально говоря вам предстоит сделать следующее:
\begin{enumerate}
\item реализовать следующие алгоритмы машинного обучения: Linear/ Logistic Regression, SVM, KNN, Naive Bayes в отдельных классах

\item Данные классы должны наследоваться от BaseEstimator и ClassifierMixin, иметь методы fit и predict 

\item Вы должны организовать весь процесс предобработки, обучения и тестирования с помощью Pipeline 

\item Вы должны настроить гиперпараметры моделей с помощью кросс валидации, вывести и сохранить эти гиперпараметры в файл, вместе с обученными моделями

\item Проделать аналогично с коробочными решениями

\item Для каждой модели получить оценки метрик:Confusion Matrix, Accuracy, Recall, Precision, ROC\_AUC curve 

\item Проанализировать полученные результаты и сделать выводы о применимости моделей

\item Загрузить полученные гиперпараметры модели и обученные модели в формате pickle на гит вместе с jupyter notebook ваших экспериментов

\end{enumerate}

\section{Ход решения}

Гуглил как реализуют модели и сравнивал с реализациями из коробки. Так как датасет несбалансированны, то некоторые модели давали не очень хороше результаты.
\section{Выводы}
По результатам видно, что наивный баес и knn показывают неплохие результаты, но knn переобучился. Собственные модели в некоторых случаях показывают достойный результат по сравнению с моделями из коробки.
\end{document}